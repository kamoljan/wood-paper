\documentclass[conference]{IEEEtran}

\usepackage{pdfpages}
\usepackage{url}

% correct bad hyphenation here
\hyphenation{op-tical net-works semi-conduc-tor}

\begin{document}
\title{Report for Wood's paper: 
The Feasibility of Magnetic Recording at 10 Terabits Per Square Inch on Conventional Media}

\author{\IEEEauthorblockN{Kamoliddin Mavlonov}
\IEEEauthorblockA{Graduate School of Science and Engineering\\
Ehime University\\
3 Bunkyou-cho Matsuyama Ehime 790-8577, Japan\\
kamol@koblab.cs.ehime-u.ac.jp}}

\maketitle

\begin{abstract}
This report is purely based on my own comprehension of this paper.
\end{abstract}

\IEEEpeerreviewmaketitle

\section{Introduction}
In 2000, Wood publishes a paper: The Feasibility of Magnetic Recording at 1 Terabits Per Square Inch~\cite{Wood2000}. It says, that conventional recording would reach a limit at around 1 Terabit/in$^2$.

However, in 2009, he admits~\cite{Wood2009} the current hard disk drive (HHD) technology is already reaching this limit.

Moreover, the Advanced Storage Technology Consortium (ASTC)~\cite{ASTC} released the 2014 roadmap for HHD area density as shown in Fig.\,\ref{fig_astc}

\begin{figure}[!hbt]
\includegraphics[height=0.25\textheight]{ASTC}
\caption{Data synchronization between two devices}
\label{fig_astc}
\end{figure}


\section{Shingled Magnetic Recording}

Shingled Magnetic Recording (SMR) is one of the latest technique of magnetic recording to increase the aerial density up to 10Tb/in$^2$. Without much change in the current design of hard disks read and write heads, SMR provides a way to achieve this increased aerial density. SMR writes and reads sequentially. It is not an optimal solution for update-in-place. For writing in SMR, the write head starts writing the track sequentially thereby overlapping the previously written track. It can be imagined as a house with roof top shingles. In SMR several tracks are grouped under a band, so in order to  perform update-in-place, bands are read intially until that point then update takes place by sequentially overriding the tracks until the end of the band. SMR can be efficiently used for the use cases which involves sequential read and write operation with less of updates. Following are some example use cases.
\itemize
\item Back ups
\item CCTV recording
\item DNA models in the medical fields
\end

Since SMR writes overlap with the previous track, it might slow down the writing process. SMR can be classified as below, which helps in mitigating the problem
\itemize
\item Drive Managed SMR - It uses a firmware to mitigate the slowness of writing.
\item Host Managed SMR - The operating system needs to know how to handle the situation based on the needs.
\end


uses perpendicular recording, which already reaching this limit.
However, alternative technologies: heat-assisted magnetic recording (HAMR)~\cite{Rottmeyer} and bit patterned media (BPM)~\cite{Terris}

\subsection{Subsection Heading Here}
Subsection text here.


\subsubsection{Subsubsection Heading Here}
Subsubsection text here.



\section{Conclusion}
The conclusion goes here.


% use section* for acknowledgment
\section*{Acknowledgment}


The authors would like to thank...

% set second argument of \begin to the number of references
% (used to reserve space for the reference number labels box)
\begin{thebibliography}{1}

\bibitem{Wood2000}
R.~Wood, \emph{The feasibility of magnetic recording at 1 terabit per square
inch}, IEEE Trans. Magn., vol. 36, pp. 36–42, Jan. 2000.

\bibitem{Rottmeyer}
R.~Rottmeyer et al., \emph{Heat-assisted magnetic recording}, IEEE Trans.
Magn., vol. 42, no. 10, pp. 2417–2421, Oct. 2006.

\bibitem{Terris}
B.~Terris, T. Thomson, and G. Hu, \emph{Patterned media for future magnetic
data storage}, Microsyst. Technol., vol. 13, no. 2, pp. 189–196, Nov.
2006.

\bibitem{Wood2009}
R.~Wood, M. Williams, A. Kavcic, J. Miles, \emph{The Feasibility of Magnetic Recording at 10 Terabits Per Square Inch on Conventional Media}, IEEE Trans. Magn., vol. 45, pp. 917-923, Feb. 2009.

\bibitem{ASTC}ASTC Technology Roadmap - 2014 v8, \url{http://idema.org/?page_id=416}, 2014.

\end{thebibliography}


% that's all folks
\end{document}