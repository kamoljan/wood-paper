\section{Introduction}
In 2000, Wood publishes a paper: The Feasibility of Magnetic Recording at 1 Terabits Per Square Inch~\cite{Wood2000}. It says, that conventional recording would reach a limit at around 1 Terabit/in$^2$.

However, in 2009, he admits~\cite{Wood2009} the current hard disk drive (HDD) technology is already reaching this limit. Wood is right that to assure continued capacity growth in HDD need alternative technologies: heat-assisted magnetic recording (HAMR)~\cite{Rottmeyer} and bit patterned media (BPM)~\cite{Terris}. 

Toward proof of the concept, the Advanced Storage Technology Consortium (ASTC)~\cite{ASTC} released the 2014 roadmap for HDD area density as shown in Fig.\,\ref{fig_astc}.

\begin{figure}[!hbt]
\includegraphics[height=0.25\textheight]{ASTC}
\caption{Data synchronization between two devices}
\label{fig_astc}
\end{figure}

Figure\,\ref{fig_astc} shows the current HDD technology is Perpendicular recording~\cite{HGST} and future HDD technologies currently in development include: HAMR and BPM with Two Dimensional Magnetic Recording (TDMR)~\cite{Krishnan} and Shingled Magnetic Recording (SMR)~\cite{Gibson}
